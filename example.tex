% Example document for LaTeX projects
%
% Copyright (c) 2007-2013 Fabian Fagerholm
%
% Copying and distribution of this file, with or without modification,
% are permitted in any medium without royalty provided the copyright
% notice and this notice are preserved. This file is offered as-is,
% without any warranty.

\documentclass{article}

\usepackage[hyphens]{url}
\usepackage[pdftex]{thumbpdf}
\usepackage[
    bookmarks=true,bookmarksnumbered=true,hypertexnames=false,
    breaklinks=true,linkbordercolor={1 1 1},pdfborder={0 0 0}]{hyperref}
\hypersetup{
    pdfauthor={Fabian Fagerholm},
    pdftitle={Example Document},
    pdfsubject={LaTeX example},
    pdfkeywords={LaTeX, example}
}
\pdfadjustspacing=1
\clubpenalty=10000
\widowpenalty = 10000

\synctex=1

\usepackage[utf8]{inputenc}
\usepackage{longtable}
\usepackage{booktabs}
\usepackage{multirow}
\usepackage{paralist}
\let\labelindent\relax
\usepackage{enumitem}
\usepackage{tikz}
\usetikzlibrary{positioning}
\usetikzlibrary{shadows}
\usetikzlibrary{patterns}
\usepackage{microtype}

\begin{document}

\title{Example Document}
\author{Fabian Fagerholm}

\maketitle

\begin{abstract}
This document is an example template for \LaTeX{} projects. It shows how to do some basic formatting of text, tables, and figures. You should modify it according to your needs.
\end{abstract}

%%%%% REMOVE %%%%%

\section{Introduction}
\label{sec:introduction}

\LaTeX{} users often collect their favourite code snippets in an example document from which they can easily copy them into actual documents. This is mine.

\section{Using this document}

You can use this document as follows:
\begin{itemize}
    \item Rename it to a name of your choosing.
    \item Change the following:
    \begin{itemize}
        \item pdfauthor
        \item pdftitle
        \item pdfsubject
        \item pdfkeywords
        \item title
        \item author
    \end{itemize}
    \item Remove the text between the ``\%\%\%\%\% REMOVE \%\%\%\%\%'' markers.
    \item Change the document class if needed.
    \item Change or remove the abstract as needed.
    \item Make any necessary adjustments that the document class requires.
\end{itemize}
You should now have a clean document which you can use for your own needs.

\subsection{Tables}

To make a nicely formatted table, use the following code:

\begin{verbatim}
\begin{table}
\caption{Example table}
\label{tab:exampletable}
\centering
\begin{tabular}{rrr}
\toprule

A & B & C \\
\midrule

1 & 44 & 67 \\
56 & 19 & 8 \\
104 & 3 & 42 \\

\bottomrule
\end{tabular}
\end{table}
\end{verbatim}

\subsection{Figures}

To include a figure, use the following code:
\begin{verbatim}
\begin{figure}
\begin{center}
\includegraphics[scale=0.5]{filename}
\end{center}
\caption{Example caption.}
\label{fig:examplefigure}
\end{figure}
\end{verbatim}

%%%%% REMOVE %%%%%

\end{document}
